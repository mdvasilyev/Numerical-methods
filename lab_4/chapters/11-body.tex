\justifying
\textbf{Цель работы:}
Научиться находить собственные значения матрицы с помощью метода Ньютона и собственные векторы с помощью итерационных методов Якоби, Зейделя и SOR метода.

\textbf{Задание:}
Вычислить собственные значения симметричной и положительно-определенной матрицы
\begin{equation}\label{matrix}
    B = \begin{pmatrix}
    16 & 3 & 2 \\
    3 & 5 & 1 \\
    2 & 1 & 10 
    \end{pmatrix}
\end{equation}
с помощью метода Ньютона. А затем вычислить соответствующие собственные векторы с помощью итерационных методов Якоби, Зейделя и SOR методом.

\textbf{Ход работы:}

Определить собственные значения матрицы можно из уравнения
\begin{equation}\label{eival_equation}
    f(\lambda) = det(B - \lambda I) = 0.
\end{equation}
Данное уравнение можно решить методом Ньютона. Тогда собственное значение на следующем шаге будет определяться по формуле:
\begin{equation}\label{newton_method}
    \lambda_i^{(n + 1)} = \lambda_i^{(n)} - \frac{f(\lambda_i^{(n)})}{f'(\lambda_i^{(n)})}, \quad i, n \in \mathbb{N}.
\end{equation}

Для нахождения собственных векторов сведем исходную задачу
\begin{equation}
    B x = \lambda x
\end{equation}
к виду
\begin{equation}
    (B - \lambda I) x = A x = 0.
\end{equation}

Представим матрицу системы $A$ в виде суммы трех матриц:
\begin{equation}
    A = B - \lambda I = A_1 + D + A_2,
\end{equation}
где $I$ -- единичная матрица, $D$ -- матрица, состоящая только из диагонали, $A_1$~-- нижняя треугольная матрица с нулями на диагонали и выше, $A_2$ -- верхняя треугольная матрица с нулями на диагонали и ниже.

С учетом такого разложения матрицы $A$, метод Якоби записывается в следующей итерационной форме:
\begin{equation}\label{Jacobi}
    x^{(n+1)} = -D^{-1}A_1x^{(n)} -D^{-1}A_2x^{(n)}
\end{equation}

Метод Зейделя записывается, как:
\begin{equation}\label{Zeidel}
    x^{(n+1)} = -(D + A_1)^{-1}A_2x^{(n)}
\end{equation}

Метод SOR получается из метода Зейделя, если введен дополнительный параметр $\omega$:
\begin{equation}\label{SOR}
    x^{(n+1)} = (I + \omega D^{-1}A_1)^{-1}[(1 - \omega)I - \omega D^{-1}A_2]x^{(n)}
\end{equation}

Реализуя указанные выше рекуррентные соотношения для каждого $\lambda_i$, получаем следующие результаты:
\begin{table}[hb]
\centering
\begin{tabular}{|l|l|l|l|} 
\hline
$\lambda_i$ & 4.193  & 9.396  & 17.411   \\ 
\hline
     & -0.231 & -0.296 & -0.927  \\
$\vec{x_i}$ & 0.969 & 0.015 & -0.247  \\
     & -0.087 & 0.955 & -0.283  \\
\hline
\end{tabular}\label{table}
\end{table}

\textbf{Выводы:}

В настоящей лабораторной работе была решена задача на собственные числа и собственные векторы матрицы~(\ref{matrix}). Собственные значения были найдены из уравнения~(\ref{eival_equation}) с помощью метода Ньютона~(\ref{newton_method}). По найденым собственным значениям были получены соответствующие собственные векторы по алгоритмам Якоби~(\ref{Jacobi}), Зейделя~(\ref{Zeidel}) и SOR~(\ref{SOR}). Результаты вычислений приведены в таблице выше.

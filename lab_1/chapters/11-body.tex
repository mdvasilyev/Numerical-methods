\justifying
\textbf{Цель работы:}
Научиться решать системы линейный алгебраических уравнений методом последовательны исключений Гаусса. Научиться производить $LU$ разложение матрицы системы.

\textbf{Задание:}
Решить систему линейных алгебраических уравнений методом последовательных исключений Гаусса.
\begin{equation}\label{task}
    Ax = f,\quad A = 
    \begin{pmatrix}
    2 & 1 & 3 \\
    11 & 7 & 5 \\
    9 & 8 & 4
    \end{pmatrix}, \quad
    f = 
    \begin{pmatrix}
    10 \\
    2 \\
    6
    \end{pmatrix}.
\end{equation}

Получить $LU$ разложение матрицы системы.

\textbf{Ход работы:}

Выполним алгоритм Гаусса для нахождения неизвестных.
\begin{enumerate}
    \item Проверим, что определитель матрицы $A$ не равен нулю:
    \begin{equation}
        \det A \neq 0.
    \end{equation}
    \item Воспользуемся общим алгоритмом, переводящим матрицу $A$ в матрицу $U$:
    \begin{equation}
        A \longrightarrow U \Longleftrightarrow 
        \begin{pmatrix}
        a_{11} & a_{12} & \dotsb & a_{1n} \\
        a_{21} & a_{22} & \dotsb & a_{2n} \\
        \dotsb & \dotsb & \dotsb & \dotsb \\
        a_{n1} & a_{n2} & \dotsb & a_{nn} \\
        \end{pmatrix}
        \longrightarrow
        \begin{pmatrix}
        1 & u_{12} & \dotsb & u_{1n} \\
        0 & 1 & \dotsb & u_{2n} \\
        \dotsb & \dotsb & \dotsb & \dotsb \\
        0 & 0 & \dotsb & 1 \\
        \end{pmatrix}.
    \end{equation}
    Получившаяся матрица $U$ называется верхнетреугольной.
    \item Уравнение $Ax = f$ перешло в уравнение $Ux = y$. Причем, коэффициенты $u_{kj}$ и $y_k$ вычисляются по следующим формулам:
    \begin{equation}
        u_{kj} = \frac{a_{kj}^{(k-1)}}{a_{kk}^{(k-1)}}, \quad 
        y_{k} = \frac{f_{k}^{(k-1)}}{a_{kk}^{(k-1)}}
    \end{equation}
    \begin{equation}
        a_{ij}^{(k)} = a_{ij}^{(k-1)} - a_{ik}^{(k-1)}u_{kj}, \quad i,j = k+1, \dotsb, n
    \end{equation}
    \begin{equation}
        f_{i}^{(k)} = f_{i}^{(k-1)} - a_{ik}^{(k-1)}y_{k}, \quad i,j = k+1, \dotsb, n
    \end{equation}
    \begin{equation}
        f_1^{(0)} = f_1, \quad a_{1j}^{(0)} = a_{1j}, \quad j = 1, 2, \dotsb, n
    \end{equation}
    \item Неизвестные $x_i$ найдем с помощью обратных подставновок по формуле:
    \begin{equation}
        x_i = y_i - \sum_{j = i + 1}^{n} u_{ij}x_j, \quad 1 \leq i \leq n-1.
    \end{equation}
\end{enumerate}

Выполним $LU$ разложение матрицы системы.
\begin{enumerate}
    \item Сделав Гауссов спуск, мы из системы уравнений $Ax = f$ получили $Ux = y$. Рассматривая соотношения между $y$ и $f$, получаем, что $f = Ly$. Причем $L$ вычисляется по формулам:
    \begin{equation}
        l_{ij} = 
        \begin{cases}
            a_{ij}^{(j-1)}, & i \geq j, \\
            0, & i < j.
        \end{cases}
    \end{equation}
    \item Так как $Ax = f$, $Ux = y$ и $Ly = f$, получаем
    \begin{equation}
        Ax = Ly = LUx.
    \end{equation}
    Следовательно, $A = LU$.
\end{enumerate}

\textbf{Результаты:}

Программа, использующая описанные выше формулы и написанная на Python, выдает следующие результаты:
\begin{equation}
    x = \begin{pmatrix}
    -3.38461538 \\
    2.11538462 \\
    4.88461538
    \end{pmatrix},
\end{equation}
\begin{equation}
    L = \begin{pmatrix}
    2 & 0 & 0 \\
    11 & 1.5 & 0 \\
    9 & 3.5 & 17.33333333
    \end{pmatrix}, \quad
    \begin{pmatrix}
    1 & 0.5 & 1.5 \\
    0 & 1 & -7.66666667 \\
    0 & 0 & 1
    \end{pmatrix}.
\end{equation}
При перемножении $L$ и $U$ получаем следующую матрицу:
\begin{equation}
    \begin{pmatrix}
    2 & 1 & 3 \\
    11 & 7 & 5 \\
    9 & 8 & 4
\end{pmatrix},
\end{equation}
которая в точности совпадает с исходной матрицей $A$.

\textbf{Выводы:}

В настоящей работе была решена решена система линейных алгебраических уравнений~(\ref{task}) с помощью метода последовательных исключений Гаусса. Также было получено $LU$ разложение матрицы системы. Лабораторная работа была выполнена на языке Python.